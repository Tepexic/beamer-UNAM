% DO NOT COMPILE THIS FILE DIRECTLY!
% This is included by main.tex

\begin{frame}[plain,t] % the tile page must always be set in a plain frame
\maketitle
\end{frame}


\begin{frame}{Table of contents}
    \tableofcontents
\end{frame}
  

\AtBeginSection[]
{
  \begin{frame}
    \frametitle{Table of contents áéíóúñü}
    \tableofcontents[currentsection]
  \end{frame}
}

%=============================================================
% What's this?
%=============================================================
\section{What's this?}
\begin{frame}{What is this?}{\LaTeX Beamer Theme}
    \begin{itemize}
    \item The main prupose of this theme is to provide a nice theme for Beamer, in the case you don't want to use the default ones.
    \item The name comes from the National Autonomous University of Mexico (UNAM)
    \item Other themes are named after locations of Universities or conferences
    \end{itemize}
\end{frame}



%=============================================================
% Theme usage
%=============================================================
\subsection{Theme usage}

\begin{frame}[fragile]{How to use the theme}
\begin{itemize}
\item Install Beamer
\item Read the Beamer documentation
  \begin{itemize}
  \item \href{http://ctan.mirrors.hoobly.com/macros/latex/contrib/beamer/doc/beameruserguide.pdf}{Beamer doc on CTAN}
  \item \href{http://web.mit.edu/rsi/www/pdfs/beamer-tutorial.pdf}{Quick start}
  \end{itemize}
\end{itemize}
\end{frame}

%=============================================================
% Theme config
%=============================================================
\begin{frame}[t,fragile]{Configuring the theme}
\begin{itemize}
\item Beamer themes can be configured with options between \verb![! and
      \verb!]!
  \begin{itemize}
  \item \verb!\usetheme[options=values]{unam}!
  \end{itemize}
\item If you do not specify any option, you get
  \begin{itemize}
  \item Simple title page
  \item Watermark theme
  \end{itemize}
\end{itemize}
\end{frame}


%=============================================================
% Theme options
%=============================================================
\begin{frame}[allowframebreaks,fragile]{Theme options}
\begin{itemize}
\item \verb! titlepagelogo = route/file.ext!
    \begin{itemize}
        \item Provides the crest for the fancy title page.
        \item The default file is \verb!unamlogo.pdf/eps!.
    \end{itemize}
\item \verb!fancytitlepage = true/false!
    \begin{itemize}
    \item \verb!fancytitlepage = true! provides a fancy title page
    \item \verb!fancytitlepage = false! provides a plain title page
    \end{itemize}
\item \verb!sidebartheme = true/false!
\begin{itemize}
    \item \verb!sidebartheme = true! provides a sidebar theme, no watermark
    \item \verb!sidebartheme = false! provides a watermark theme, with navigation on the header
    \end{itemize}
\end{itemize}
\end{frame}


\subsection{Theme features}
%=============================================================
% Alternative title page
%=============================================================
\begin{frame}[t,fragile]{Fancy title page}
\begin{itemize}
\item A fancy title page can be enabled with the \verb!fancytitlepage = true! option
\item You can put a logo in the title page, just pass the file name using the
      \verb! titlepagelogo = route/file.ext! option.
\item Remember to use a \textbf{plain} and \textbf{top-aligned} frame when using fancy title
      pages:\\
      \vskip1ex
      \verb!\begin{frame}[t,plain]!\\
      \verb!\titlepage!\\
      \verb!\end{frame}!
\end{itemize}
\end{frame}


%=============================================================
%  sidebar theme
%=============================================================

\begin{frame}[t,fragile]{Sidebar theme}
    \begin{itemize}
        \item \verb!sidebartheme = true!
        \item This theme has the crest on the left sidebar, and also includes navigation.
        \item Institute, date, author and pages are on the footer.
        \item When this option is set to \verb!false!, a watermark theme is used.
    \end{itemize}
\end{frame}


%=============================================================
%  Dummy text
%=============================================================
\section{Dummy section}
\subsection{Dummy subsection}
\begin{frame}{Frame Title}
\begin{theorem}
There is no largest prime number.
\end{theorem}
\end{frame}

\begin{frame}{Frame Title}
\begin{proof}
\begin{enumerate}
\item<1-| alert@1> Suppose $p$ were the largest prime number.
\item<2-> Let $q$ be the product of the first $p$ numbers.
\item<3-> Then $q+1$ is not divisible by any of them.
\item<4-> But $q + 1$ is greater than $1$, thus divisible by some prime
number not in the first $p$ numbers.\qedhere
\end{enumerate}
\end{proof}
\end{frame}


\begin{frame}{Frame Title}
\blinddescription
\end{frame}

\subsection{Dummy subsection}
\begin{frame}[allowframebreaks]{Frame Title}
\blindmathpaper
\end{frame}

\subsubsection{Dummy subsubsection}
\begin{frame}{Frame Title}
\blindlist{itemize}[4]
\end{frame}

\begin{frame}{Frame Title}
\blinddescription
\end{frame}

\section{Dummy section}
\subsection{Dummy subsection}
\begin{frame}{Frame Title}
\blindenumerate
\end{frame}

\begin{frame}{Frame Title}
\blindlist{itemize}[4]
\end{frame}

\begin{frame}{Frame Title}
\blinddescription
\end{frame}

\subsection{Dummy subsection}
\begin{frame}[allowframebreaks]{Frame Title}
\blindmathpaper
\end{frame}